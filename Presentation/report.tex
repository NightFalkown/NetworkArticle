\documentclass[compress]{beamer}

\usepackage[utf8]{inputenc}
\usepackage{amssymb}
\setbeamertemplate{caption}{\raggedright\insertcaption\par}

\usetheme[navigation]{UMONS}
%\usetheme[navigation, no-subsection, no-totalframenumber]{UMONS}

\newcommand{\IR}{\mathbb{R}}


\title{Bandwidth estimation : metrics, measurement techniques, and tools}
\author{M. Lempereur \\
J. Gheysen}
\institute[(Info)]{%
  Département d'Informatique\\
  Université de Mons
  \\[2ex]
  \includegraphics[height=4ex]{UMONS}\hspace{2em}%
  \raisebox{-1ex}{\includegraphics[height=6ex]{UMONS_FS}}
}

\begin{document}

\begin{frame}[plain]
  \titlepage
\end{frame}

\begin{frame}
  \tableofcontents
\end{frame}

\section{Introduction}
\begin{frame}{Introduction}

\end{frame}

%%%%%%%%%%%%%%%%%%%%%%%%%%%%%%%%%%%%%%%%%%%%%%%%%%%%%%%%%%%%%%
%Dans l'utilisation quotidienne beaucoup de termes sont
%confondus, le but ici est de formaliser la définition autour
%de la vitesse d'une connection et de la "bande passante".
%D'apporter des mesures précises et de faire le point sur les
%technologies existantes à ce niveau. 
%%%%%%%%%%%%%%%%%%%%%%%%%%%%%%%%%%%%%%%%%%%%%%%%%%%%%%%%%%%%%%

\begin{frame}{Introduction}{Vocabulaire 1/2}
	\begin{figure}
   		\begin{minipage}[c]{.46\linewidth}
   			\includegraphics[scale=0.23]{capacity.JPG}
   			\caption{Capacité}
   		\end{minipage} \hfill
   		\begin{minipage}[c]{.46\linewidth}
      		\includegraphics[scale=0.18]{bande_passante.jpg}
      		\caption{Bande Passante}
   		\end{minipage}
   		\begin{minipage}[c]{.46\linewidth}
   			\includegraphics[scale=0.2]{tcp.png}
   			\caption{Bulk-Transfert-Capacity}
   		\end{minipage} \hfill
	\end{figure}
\end{frame}

\begin{frame}{Introduction}{Vocabulaire 2/2}
\begin{itemize}
	\item Différenciation entre liens de la couche données (1) et couche IP(2).
	\begin{enumerate}
		\item Segment : Lien physique point à point. %Etendre aux autres points dans l'article ?
		\item Hop (saut) : Consiste en un ensemble de segments. %Connectés à travers switchs/bridges...
	\end{enumerate}
	\item Chemin end-to-end : Lie un hôte à une source via une suite de hops.
	\item Que mesurer ? Dans quelle situation ? 
	\begin{itemize}
		\item Lien unique (hop) : Capacité, Bande Passante.
		\item Chemin : Capacité, Bande Passante, Bulk-Transfert-Capacity.
	\end{itemize}
\end{itemize}
\end{frame}

%%%%%%%%%%%%%%%%%%%%%%%%%%%%%%%%%%%%%%%%%%%%%%%%%%%%%%%%%%%%%%
%Tout d'abord != entre lien et chemin (= ensemble de liens)
%Ici nous utilisons lien pour donner définition globale.
%Capacité : Valeur maximale du flux pouvant traverser un lien,
%càd quantité maximale de bits pouvant transiter à travers
%le lien par unité de temps. 
%Bande passante disponible : Capacité du lien non utilisée,
%alors utilisable afin d'écouler du traffic. 
%BTC : Représente le débit atteignable par une seule 
%connexion TCP. 
%%%%%%%%%%%%%%%%%%%%%%%%%%%%%%%%%%%%%%%%%%%%%%%%%%%%%%%%%%%%%%


\section{Données mesurées}
\subsection{Capacité}
\begin{frame}{Capacité}

\end{frame}
\subsection{Bande Passante}
\begin{frame}{Bande Passante}

\end{frame}
\subsection{Bulk-Transfert-Capacity}
\begin{frame}{Bulk-Transfert-Capacity (BTC)}

\end{frame}

\section{Techniques de mesure}
\begin{frame}{Techniques de mesure}

\end{frame}

\section{Taxonomie des outils de mesure}
\begin{frame}{Taxonomie des outils de mesure}

\end{frame}


\end{document}
%%% Local Variables: 
%%% mode: latex
%%% TeX-master: t
%%% End: 
\grid
